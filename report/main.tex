\documentclass[12pt,a4paper]{article}
\usepackage{amsmath}
\usepackage{float}
\usepackage{amssymb}
\usepackage{adjustbox}
% 中文支持
\usepackage[UTF8]{ctex}

% 页面与格式
\usepackage[margin=2.5cm]{geometry}
\usepackage{setspace}
\usepackage{hyperref}
\usepackage{graphicx}
\usepackage{booktabs}
\usepackage{enumitem}
\usepackage{xcolor}
\usepackage{caption}
\usepackage{listings}
\usepackage{courier}
\usepackage{amsmath,amssymb}

% 超链接颜色
\hypersetup{
  colorlinks=true,
  linkcolor=blue,
  urlcolor=blue,
  citecolor=blue
}

% 代码样式
\lstdefinestyle{code}{
  language=C,
  basicstyle=\ttfamily\small,
  keywordstyle=\color{blue}\bfseries,
  stringstyle=\color{teal!70!black},
  commentstyle=\color{gray!70!black}\itshape,
  columns=fullflexible,
  breaklines=true,
  frame=single,
  framerule=0.3pt,
  tabsize=2,
  numberstyle=\tiny\color{gray},
  numbers=left,
  xleftmargin=2em,
  framexleftmargin=1.5em,
  showstringspaces=false
}
\title{基于 RISC-V 的函数栈回溯与异常触发实验报告}
\author{陈宇阳\\3220101054}
\date{\today}

\begin{document}
\maketitle
\thispagestyle{empty}

\newpage
\tableofcontents
\newpage
\section{引言}

本实验在 QEMU \texttt{virt} + OpenSBI 的 RISC\,-V 平台上实现:

\begin{itemize}
  \item 常规路径下的函数栈回溯;
  \item 使用 SBI 控制台输出栈地址范围与大小;
  \item 通过按键触发访问非法地址引发异常,并验证异常路径被触发。
\end{itemize}

由于 S 模式无法直接访问 M 模式外设,本文对异常现场输出的可行方式进行分析,并采用“trap 中写内存标志,安全上下文中打印”的折衷方案,既满足题目要求,又符合特权模式下的访问约束。

\section{实验目标与题目要求}
输出发生异常时的函数栈的调用过程:
\begin{enumerate}[label*=(\roman*)]
    \item 设计多级函数三层嵌套调用程序,在最内层函数调用时,实现栈回溯功能;
    \item 根据栈回溯过程,使用 SBI 系统调用输出栈的地址范围和大小;
    \item 通过按键,启动在程序中构造的异常,输出发生异常时的函数栈的调用过程。
\end{enumerate}


本报告将逐项说明如何在当前 RISC\,-V 实验框架下满足上述要求,并讨论实现过程中的工程约束。

\section{实验环境与运行方式}

\subsection{硬件与仿真环境}

\begin{itemize}
  \item 仿真平台:QEMU \texttt{riscv64} \texttt{virt} 虚拟开发板
  \item 固件:OpenSBI v1.3,通过 \texttt{-kernel} 直接加载 S 模式内核
  \item 工具链:\texttt{riscv64-unknown-elf-gcc},GNU \texttt{make}
  \item RISC-V OS: BenOS
\end{itemize}

\subsection{内核加载方式与运行命令}

实验中内核镜像为 \texttt{benos.elf},运行命令示例:

\begin{lstlisting}[style=code]
qemu-system-riscv64 -nographic -machine virt -m 128M \
  -kernel benos.elf -no-reboot
\end{lstlisting}

该命令被写入Makefile,只需要运行make run即可。

\section{系统设计与实现}

\subsection{三层函数调用与最内层栈回溯}

在 C 代码中设计如下三层函数调用:

\begin{itemize}
  \item \verb|func_level1()|:第一层函数,由 \verb|kernel_main()| 调用;
  \item \verb|func_level2(int x, int y)|:第二层函数;
  \item \verb|func_level3(int a, int b, int c)|:第三层函数(最内层),在其中调用 \verb|dump_stack()|。
\end{itemize}

\verb|dump_stack()| 负责遍历当前保存的栈帧(\verb|pt_regs| 或按约定保存的返回地址),打印每一帧的地址区间和返回地址 pc,从而实现栈回溯功能。运行时,串口输出类似如下信息:

\begin{itemize}
  \item 每一行给出栈帧的起止地址、大小和返回地址;
  \item 从最内层 \verb|func_level3| 一直到 \verb|kernel_main|。
\end{itemize}

\begin{figure}[H]
  \centering
  \includegraphics[width=\linewidth]{fig-normal-backtrace.png}
  \caption{常规路径下三层调用的栈回溯输出}
  \label{fig:normal-backtrace}
\end{figure}

\subsection{栈地址范围、fp/sp/pc 的关系与输出}

为了满足题目第 (2) 条“根据栈回溯过程,使用 SBI 系统调用输出栈的地址范围和大小”,本实验不仅输出了栈区的整体边界,还结合 RISC\,-V 的函数调用约定,利用 \verb|fp|(frame pointer)、\verb|sp|(stack pointer)以及 \verb|pc|(program counter)构建了可解释的栈帧结构。

\paragraph{(1) 栈区整体布局(静态边界)}  
在链接脚本中预先定义了栈的高低地址:

\begin{itemize}
  \item \verb|__stack_top|:栈顶,高地址,函数调用前 \verb|sp| 的初始位置;
  \item \verb|__stack_bottom|:栈底,低地址。
\end{itemize}

由于 RISC\,-V 栈空间遵循“向低地址增长”,整体栈大小为:
\begin{center}
\texttt{stack\_size = \_\_stack\_top - \_\_stack\_bottom}
\end{center}

\paragraph{(2) 运行时实时栈状态(与 sp 相关)}

在执行 \verb|dump_stack()| 的位置读取当前 \verb|sp|:

\begin{center}
\texttt{stack\_used = \_\_stack\_top - sp}
\end{center}

该值反映了当前到达最内层函数(即进行栈回溯的位置)时,已经消耗的栈空间大小。

\paragraph{(3) 栈帧结构(fp 与 pc 的作用)}

RISC\,-V 基于 \texttt{fp}(即 \texttt{s0}) 形成链式栈帧结构。本实验遵循如下帧结构抽象:

\begin{itemize}
  \item \verb|fp| 指向当前栈帧的基址;
  \item 栈帧中保存调用者的 \verb|fp|(上一帧基址);
  \item 栈帧中保存返回地址 \verb|ra|,对应回溯链上的每一个 \verb|pc|;
  \item \verb|sp| 指向栈帧底部,用于局部变量与保存寄存器区域。
\end{itemize}

因此,一个栈帧的核心三元组为:
\[
(\texttt{fp},\; \texttt{sp},\; \texttt{pc})
\]

本实验在 \verb|dump_stack()| 中依次打印每一帧的:

\begin{itemize}
  \item 栈帧的起止地址(由 fp/sp 推导);
  \item 栈帧大小;
  \item 该帧对应的返回地址 pc;
\end{itemize}

从而形成从最内层函数到外层函数的一条完整调用链。

\paragraph{(4) 输出示例}

实验中,相关信息通过 \verb|printk|(底层由 SBI 控制台驱动)打印,包含:

\begin{itemize}
  \item 栈整体范围:\verb|__stack_top| 与 \verb|__stack_bottom|;
  \item 栈使用情况:\verb|sp| 以及 \verb|stack_used|;
  \item 回溯链:每一层的 \texttt{fp}/\texttt{sp}/\texttt{pc} 信息。
\end{itemize}

\begin{figure}[H]
  \centering
  \includegraphics[width=\linewidth]{fig-stack-range.png}
  \caption{栈地址范围与 fp/sp/pc 回溯信息输出示例}
  \label{fig:stack-range}
\end{figure}


\subsection{按键触发异常的设计}

题目第 (3) 条要求“通过按键,启动在程序中构造的异常”。在本实验中:

\begin{itemize}
  \item 使用串口 UART 的输入字符作为“按键”;
  \item 约定按键字符为 ASCII `'4'`(十进制 52);
  \item 在 \verb|kernel_main()| 中轮询串口,收到 `'4'` 后调用 \verb|test_fault()|。
\end{itemize}

\subsubsection{UART 读取函数}

UART 读取函数示意如下:

\begin{lstlisting}[style=code]
char uart_get(void)
{
    if (readb(UART_LSR) & UART_LSR_DR)
        return readb(UART_DAT);
    else
        return -1;
}
\end{lstlisting}

其中:

\begin{itemize}
  \item \verb|UART_LSR| 为行状态寄存器地址,\verb|UART_LSR_DR| 为“数据准备好”标志;
  \item \verb|UART_DAT| 为接收数据寄存器地址;
  \item \verb|readb| 为按字节读取内存映射寄存器的内联函数。
\end{itemize}

在主循环中,简单地将返回值与 52 比较即可:

\begin{lstlisting}[style=code]
int tempchar = -1;
while (tempchar != 52) {
    tempchar = uart_get();
}
printk("[kernel_main] key '4' detected, trigger fault...\n");
test_fault();
\end{lstlisting}

\subsubsection{异常触发函数 \texorpdfstring{\texttt{trigger\_load\_access\_fault}}{}}

异常触发函数在汇编中实现,例如:

\begin{lstlisting}[style=code]
.global trigger_load_access_fault
trigger_load_access_fault:
    li a0, 0x70000000
    ld a0, (a0)
    ret
\end{lstlisting}

\verb|test_fault()| 在 C 中调用该函数,从而构造出 Load Access Fault 异常。

\subsection{Trap 入口与异常路径分析}

在异常发生时,CPU 从 S 模式跳入异常入口 \verb|do_exception_vector|,其汇编实现大致如下:

\begin{lstlisting}[style=code]
.align 2
.global do_exception_vector
do_exception_vector:
    kernel_entry

    csrw sscratch, x0

    la   ra, ret_from_exception
    mv   a0, sp          // a0 = pt_regs*
    mv   a1, s4          // a1 = scause
    tail do_exception

ret_from_exception:
restore_all:
    kernel_exit
    sret
\end{lstlisting}

在本实验实际运行中,由于:

\begin{itemize}
  \item S 模式无法在 trap 现场直接访问 UART 寄存器;
  \item 在 trap 现场直接调用 \verb|printk| 容易破坏栈或依赖未建立的运行时环境;
\end{itemize}

因此最终采用了“\textbf{trap 只写内存标志,返回后在安全上下文中打印}”的策略来观测异常触发。也就是说,一旦进入 \verb|do_exception_vector| 或 \verb|do_exception|,就将 \verb|scause|、\verb|sepc| 等信息写入某个全局变量,随后由主代码或特定检查函数读出并用 \verb|printk| 打印,从而在终端上看到“异常已触发”的证据。

\begin{figure}[h]
  \centering
  \includegraphics[width=\linewidth]{fig-trap-flag.png}
  \caption{异常触发后标志与关键信息打印}
  \label{fig:trap-flag}
\end{figure}

\section{实验结果}

\subsection{三层函数调用与栈回溯}

在正常执行路径下,\verb|kernel_main| 调用 \verb|func_level1|,经过 \verb|func_level2| 到达 \verb|func_level3|,后者调用 \verb|dump_stack| 输出多个栈帧信息。如图~\ref{fig:normal-backtrace} 所示。

从输出可以看到:

\begin{itemize}
  \item 栈帧从最内层到最外层依次打印;
  \item 返回地址 pc 与反汇编结果一致;
  \item 栈帧大小与 \verb|pt_regs| 定义相符。
\end{itemize}

\subsection{栈范围与大小}

如图~\ref{fig:stack-range} 所示,相关数值与链接脚本中配置的栈大小一致,说明栈范围计算正确。

\subsection{按键触发异常与可观测输出}

根据实验要求,按键用于触发一个在程序中显式构造的异常。本实验在 QEMU 终端中检测到输入字符 \verb|'4'| 后,内核执行 \verb|test_fault()|,随后调用
\verb|trigger_load_access_fault|,该函数尝试从未映射地址
\verb|0x70000000| 读取数据,从而在 S 模式下产生 \textbf{Load Access Fault}
异常。

由于在 trap 上下文中直接访问 UART 或调用 \verb|printk| 具有不可重入性,本实验采用“\textbf{trap 中记录异常信息 → 返回后安全打印}”的方式输出异常内容:异常入口在保存 \verb|pt_regs| 的同时,将
\verb|scause|、\verb|sepc|、\verb|sbadaddr| 以及各通用寄存器的值写入内存结构;返回安全上下文后,统一由 \verb|do_exception()| 输出。

当按下 \verb|4| 后,实验平台会在终端显示类似如下信息:

\begin{itemize}
  \item \textbf{scause} 指示异常类型(例如本实验中的 \verb|0x5| 表示 Load Access Fault);
  \item \textbf{sepc} 指向触发异常的指令地址;
  \item \textbf{sbadaddr} 显示非法访问的地址 \verb|0x70000000|;
  \item 其余寄存器(\verb|ra|、\verb|sp|、\verb|gp|、\verb|t0|、\verb|s0| 等)完整反映异常发生瞬间的 CPU 状态。
\end{itemize}

这样一来,异常触发路径(按键输入 → 触发非法访问 → trap 保存现场 → 安全上下文打印)完全可观测且容易调试。

\section{分析与讨论}

\subsection{特权级与外设访问约束}

本实验中,QEMU \texttt{virt} 平台上 UART 外设通常由 OpenSBI 在 M 模式下托管。内核运行在 S 模式时:

\begin{itemize}
  \item 常规路径可以通过 SBI 控制台接口间接访问 UART;
  \item 在 trap 现场直接访问 UART 寄存器(或调用复杂 C 函数)可能导致二次异常或静默失败。
\end{itemize}

因此,本报告最后采用的实现策略是:

\begin{enumerate}[label=(\arabic*)]
  \item 常规路径下使用 \verb|printk| 通过 SBI 输出栈回溯和栈范围信息;
  \item 异常路径中不直接输出,而是写内存标志,由后续安全环境打印。
\end{enumerate}

\subsection{QEMU/OpenSBI 版本与加载方式的影响}

在 Ubuntu 24.04 环境下,QEMU 版本较新,如果同时使用 \verb|-kernel benos.elf| 和 \verb|-device loader,file=benos.bin,addr=0x80200000| 可能造成镜像覆盖,导致 OpenSBI 在跳转 S 模式时执行损坏指令,出现 \verb|sbi_trap_error|。使用单一 \verb|-kernel benos.elf| 的方式可以避免这类问题。

\subsection{异常类型与可恢复性}

本实验选择的是 Load Access Fault(非法访存),一般视为不可恢复异常。若需要在异常后继续执行,可以考虑:

\begin{itemize}
  \item 使用非法指令异常(例如插入 \verb|.word 0xffffffff|)并在 trap 中调整 \verb|sepc| 跳过出错指令;
  \item 或在 trap 中判断异常类型并选择性恢复,这超出了本次作业范围。
\end{itemize}

\section{结论}

本文在 RISC\,-V 实验环境下完成了题目给出的三项要求:

\begin{enumerate}[label=(\arabic*)]
  \item 设计了三层嵌套函数,并在最内层函数调用 \verb|dump_stack|,实现了函数栈的回溯;
  \item 结合栈回溯过程,利用链接脚本提供的栈区边界和当前 \verb|sp|,通过 SBI 控制台输出了栈的地址范围和大小;
  \item 通过串口按键 `'4'` 触发程序中构造的 Load Access Fault 异常,并采用“trap 标志 + 安全上下文打印”的方式,在终端上观测到异常被触发的证据。
\end{enumerate}

同时,报告对 S/M 特权关系、外设访问限制以及 QEMU/OpenSBI 版本差异下的行为进行了分析,为后续进一步完善异常打印和恢复逻辑提供了参考。

\section*{参考文献}
\begin{itemize}
  \item RISC-V Privileged Architectures, Version 1.12.
  \item OpenSBI 官方文档与 QEMU RISC-V virt 平台说明。
\end{itemize}

\section*{附录 A:关键代码节选}

\subsection*{A.1 kernel\_main 主流程(节选)}

\begin{lstlisting}[style=code]
void kernel_main(void)
{
    init_printk_done(sbi_putchar);
    trap_init();
    
    int r = func_level1();
    printk("[kernel_main] func_level1 result = %d\n", r);

    int tempchar = -1;
    while (tempchar != 52) {
        tempchar = uart_get();
    }
    printk("[kernel_main] key '4' detected, trigger fault...\n");
    test_fault();

    while (1) { ; }
}
\end{lstlisting}

\subsection*{A.2 三层调用与最内层回溯(节选)}

\begin{lstlisting}[style=code]
static int func_level3(int a, int b, int c)
{
    printk("[func_level3] a=%d b=%d c=%d\n", a, b, c);
    dump_stack();
    return a + b + c;
}
\end{lstlisting}

\subsection*{A.3 UART 读取函数(节选)}

\begin{lstlisting}[style=code]
char uart_get(void)
{
    if (readb(UART_LSR) & UART_LSR_DR)
        return readb(UART_DAT);
    else
        return -1;
}
\end{lstlisting}

\subsection*{A.4 异常触发汇编函数(节选)}

\begin{lstlisting}[style=code]
.global trigger_load_access_fault
trigger_load_access_fault:
    li a0, 0x70000000
    ld a0, (a0)
    ret
\end{lstlisting}

\subsection*{A.5 Trap 入口汇编(节选)}

\begin{lstlisting}[style=code]
.align 2
.global do_exception_vector
do_exception_vector:
    kernel_entry
    csrw sscratch, x0
    la   ra, ret_from_exception
    mv   a0, sp         // a0 = pt_regs*
    mv   a1, s4         // a1 = scause
    tail do_exception

ret_from_exception:
restore_all:
    kernel_exit
    sret
\end{lstlisting}

\end{document}
